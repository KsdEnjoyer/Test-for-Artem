\documentclass[a4paper,12pt]{article}

\usepackage[T2A]{fontenc}
\usepackage[utf8]{inputenc}
\usepackage[russian]{babel}

\usepackage{amsmath, amssymb, amsfonts}
\usepackage{graphicx}
\usepackage{hyperref}
\usepackage{xcolor}
\usepackage{listings}
\usepackage{fancyvrb} % Улучшенное отображение кода
\usepackage{geometry} % Настройка полей
\usepackage{indentfirst} % Отступ у первого абзаца
\usepackage{tocbibind} % Добавляет содержание в оглавление

\geometry{left=2.5cm, right=2.5cm, top=2.5cm, bottom=2.5cm}

\title{\textbf{Лекция 2: Система типов в C}}
\author{}
\date{}

\lstset{
    language=C,
    basicstyle=\ttfamily\small,
    keywordstyle=\color{blue},
    commentstyle=\color{gray},
    stringstyle=\color{red},
    numbers=left,
    numberstyle=\tiny,
    stepnumber=1,
    breaklines=true,
    frame=single
}

\begin{document}

\tableofcontents % Создание оглавления

\maketitle

\section{Общие сведения о языке C}
\begin{itemize}
    \item C — регистрозависимый язык программирования.
    \item Разрядность процессора обычно соответствует размеру указателя (адреса).
\end{itemize}

\section{Система типов в C}
Типы данных в C можно разделить на несколько категорий: целочисленные, дробные и специализированные.

\subsection{Целочисленные типы}
\begin{itemize}
    \item \textbf{int} — целочисленный тип (гарантированный диапазон $[-32767, 32767]$, но фактически 32-битный на большинстве платформ).
    \item \textbf{short} — 16-битное целое число.
    \item \textbf{long} — 32-битное или 64-битное число в зависимости от компилятора и ОС.
    \item \textbf{long long} — 64-битное целое число.
    \item \textbf{char} — 8 бит, диапазон может быть знаковым (-128..127) или беззнаковым (0..255).
    \item \textbf{\_Bool} — логический тип (занимает 8 бит, но хранит 0 или 1).
\end{itemize}

Дополнительно к этим типам можно использовать модификатор \textbf{unsigned} для хранения только положительных значений.

\subsection{Дополнительные целочисленные типы}
Эти типы требуют подключения заголовочных файлов:
\begin{itemize}
    \item \textbf{size\_t} и \textbf{ptrdiff\_t} (\texttt{\#include <stddef.h>}) — типы, соответствующие разрядности кода.
    \item \textbf{uintX\_t} (\texttt{\#include <stdint.h>}) — гарантированная битность: 8, 16, 32, 64 бит.
    \item \textbf{(unsigned) \_BitInt(X)} (\texttt{\#include <stdalign.h>}) — гарантированная длина X бит.
\end{itemize}

\subsection{Форматный ввод/вывод целых чисел}
\begin{lstlisting}
printf("%d", x);    // d - Signed integer (int)
printf("%u", x);    // u - Unsigned integer (unsigned int)
printf("%x", x);    // x - Hexadecimal (lowercase letters)
printf("%X", x);    // X - Hexadecimal (uppercase letters)
printf("%hd", x);   // hd - Short integer (short)
printf("%hhd", x);  // hhd - Character (char)
printf("%ld", x);   // ld - Long integer (long)
printf("%lld", x);  // lld - Long long integer (long long)
\end{lstlisting}

\subsection{Дробные типы}
\begin{itemize}
    \item \textbf{float} — одинарная точность IEEE-754 ($10^{\pm38}$), 32 бита.
    \item \textbf{double} — двойная точность, 64 бита.
    \item \textbf{long double} — 128 бит (или 80 бит на некоторых системах), не рекомендуется.
\end{itemize}

Форматный ввод/вывод дробных чисел:
\begin{lstlisting}
printf("%f", x);   // float
printf("%lf", y);  // double
\end{lstlisting}

\subsection{Комплексные числа}
\begin{lstlisting}
#include <complex.h>
_Complex float x = 1.5f - 2.3if;
\end{lstlisting}

\section{Дополнительные возможности}
Создание пользовательских типов через \textbf{typedef}:
\begin{lstlisting}
typedef unsigned int uint;
uint x;
\end{lstlisting}

\section{Особенности констант}
\begin{itemize}
    \item Отрицательные числа в C — это представление со знаком, но отрицательных констант не существует.
    \item Ведущий 0 указывает на восьмеричное число.
    \item Префикс \texttt{0b} используется для двоичных констант.
\end{itemize}

\section{Заголовочные файлы}

В языке C существуют различные стандартные заголовочные файлы, которые подключаются для расширения возможностей работы с типами данных, математическими операциями и многими другими аспектами. Вот основные из них, использованные в данном конспекте:

\begin{itemize}
    \item \textbf{\#include <stddef.h>} — этот заголовочный файл включает определения для стандартных типов данных и макросов, таких как \texttt{size\_t}, \texttt{ptrdiff\_t} и \texttt{NULL}.
    \item \textbf{\#include <stdint.h>} — определяет типы с фиксированным размером, такие как \texttt{int32\_t}, \texttt{uint64\_t} и другие.
    \item \textbf{\#include <complex.h>} — включает поддержку комплексных чисел в языке C, позволяя работать с комплексными числами через типы \texttt{\_Complex float}, \texttt{\_Complex double}.
    \item \textbf{\#include <stdbool.h>} — добавляет поддержку логического типа \texttt{bool} и значений \texttt{true} и \texttt{false}. Этот заголовочный файл используется для работы с булевыми значениями, что значительно упрощает код.
    \item \textbf{\#include <math.h>} — предоставляет математические функции, такие как \texttt{sin}, \texttt{cos}, \texttt{sqrt} и другие.
    \item \textbf{\#include <stdio.h>} — предоставляет функции ввода/вывода, такие как \texttt{printf}, \texttt{scanf}, \texttt{fopen} и другие для работы с текстовыми и бинарными файлами.
\end{itemize}

\end{document}
