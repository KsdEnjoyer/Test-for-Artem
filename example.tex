\documentclass[a4paper,12pt]{article}
\usepackage[T2A]{fontenc}
\usepackage[utf8]{inputenc}
\usepackage[russian]{babel}
\usepackage{amsmath,amssymb,amsthm}
\usepackage{graphicx}
\usepackage{geometry}
\usepackage{hyperref}
\usepackage{titlesec}
\usepackage{enumitem}
\usepackage{tabularx}
\usepackage{booktabs}
\usepackage{longtable}
\usepackage{multicol}
\usepackage{listings}
\usepackage{xcolor}

\geometry{left=3cm,right=2cm,top=2cm,bottom=2cm}

% Настройка стиля разделов
\titleformat{\section}{\normalfont\Large\bfseries}{\thesection}{1em}{}
\titleformat{\subsection}{\normalfont\large\bfseries}{\thesubsection}{1em}{}

% Настройка листингов кода
\lstset{
    basicstyle=\ttfamily\small,
    keywordstyle=\color{blue},
    commentstyle=\color{green},
    stringstyle=\color{red},
    numbers=left,
    numberstyle=\tiny\color{gray},
    stepnumber=1,
    numbersep=5pt,
    backgroundcolor=\color{white},
    frame=single,
    rulecolor=\color{black},
    tabsize=4,
    captionpos=b,
    breaklines=true,
    breakatwhitespace=true,
    showspaces=false,
    showstringspaces=false,
    showtabs=false
}

\begin{document}

\title{Образец крупного документа LaTeX}
\author{Иван Иванов \\ Университет Примеров}
\date{\today}
\maketitle

\begin{abstract}
Это аннотация к документу. Здесь кратко описывается содержание работы, основные результаты и выводы. Аннотация обычно занимает не более одного абзаца и должна быть информативной.
\end{abstract}

\tableofcontents

\section{Введение}
\label{sec:intro}

Это вводный раздел документа. Здесь описывается актуальность темы, цели и задачи работы. Введение должно дать читателю общее представление о документе.

\subsection{Актуальность исследования}
Актуальность исследования обусловлена возрастающей потребностью в качественных LaTeX-документах для научных публикаций.

\subsection{Цели и задачи}
Основной целью данного документа является демонстрация возможностей LaTeX для создания крупных структурированных документов.

\section{Обзор литературы}
\label{sec:literature}

В этом разделе приводится обзор существующих работ по теме. Можно использовать цитирование \cite{lamport94}.

\begin{table}[h]
\centering
\caption{Сравнение систем верстки}
\label{tab:comparison}
\begin{tabularx}{\linewidth}{lXXX}
\toprule
Система & Преимущества & Недостатки \\
\midrule
LaTeX & Высокое качество верстки, поддержка сложных формул & Сложность изучения \\
Word & Простота использования & Плохая работа с формулами \\
Markdown & Простота, читаемость & Ограниченные возможности \\
\bottomrule
\end{tabularx}
\end{table}

\section{Методология}
\label{sec:methodology}

В этом разделе описываются методы, использованные в работе.

\subsection{Математические методы}
Используем уравнение второго порядка:
\begin{equation}
\label{eq:quadratic}
ax^2 + bx + c = 0
\end{equation}

Решение уравнения \eqref{eq:quadratic}:
\begin{equation}
x = \frac{-b \pm \sqrt{b^2 - 4ac}}{2a}
\end{equation}

\subsection{Программные методы}
Пример кода на Python:

\begin{lstlisting}[language=Python,caption=Пример кода]
def quadratic(a, b, c):
    """Решение квадратного уравнения"""
    discriminant = b**2 - 4*a*c
    if discriminant < 0:
        return None
    x1 = (-b + discriminant**0.5) / (2*a)
    x2 = (-b - discriminant**0.5) / (2*a)
    return x1, x2
\end{lstlisting}

\section{Результаты}
\label{sec:results}

Результаты представлены на рисунке \ref{fig:sample}.

\begin{figure}[h]
\centering
\includegraphics[width=0.8\linewidth]{example-image}
\caption{Пример изображения}
\label{fig:sample}
\end{figure}

\subsection{Таблицы с результатами}

\begin{longtable}{lrrr}
\caption{Детальные результаты экспериментов} \\
\toprule
Параметр & Серия 1 & Серия 2 & Серия 3 \\
\midrule
\endfirsthead
\multicolumn{4}{c}{Продолжение таблицы \ref{tab:long}} \\
\toprule
Параметр & Серия 1 & Серия 2 & Серия 3 \\
\midrule
\endhead
\bottomrule
\multicolumn{4}{r}{Продолжение следует...} \\
\endfoot
\bottomrule
\endlastfoot
Температура & 23.4 & 24.1 & 25.7 \\
Влажность & 45 & 47 & 43 \\
Скорость & 12.3 & 11.8 & 13.2 \\
Давление & 101.3 & 101.5 & 101.1 \\
Масса & 0.45 & 0.47 & 0.43 \\
Объем & 1.2 & 1.3 & 1.1 \\
Плотность & 0.375 & 0.362 & 0.391 \\
Вязкость & 1.23 & 1.25 & 1.21 \\
\label{tab:long}
\end{longtable}

\section{Обсуждение}
\label{sec:discussion}

В этом разделе анализируются полученные результаты и их соответствие поставленным задачам.

\subsection{Ограничения исследования}
\begin{itemize}
\item Ограниченный объем данных
\item Использование упрощенных моделей
\item Ограниченные вычислительные ресурсы
\end{itemize}

\subsection{Перспективы развития}
\begin{enumerate}
\item Расширение набора данных
\item Учет дополнительных параметров
\item Оптимизация алгоритмов
\end{enumerate}

\section{Заключение}
\label{sec:conclusion}

В работе продемонстрированы основные возможности LaTeX для создания крупных структурированных документов. Показаны примеры математических формул, таблиц, изображений и листингов кода.

\appendix
\section{Дополнительные материалы}
\label{sec:appendix}

\subsection{Дополнительные таблицы}

\begin{table}[h]
\centering
\caption{Дополнительные параметры}
\label{tab:additional}
\begin{tabular}{lr}
\toprule
Параметр & Значение \\
\midrule
Коэффициент A & 1.234 \\
Коэффициент B & 5.678 \\
Константа C & 9.101 \\
\bottomrule
\end{tabular}
\end{table}

\subsection{Дополнительные формулы}

Формула Эйлера:
\begin{equation}
e^{i\pi} + 1 = 0
\end{equation}

\bibliographystyle{unsrt}
\bibliography{references}

\end{document}
